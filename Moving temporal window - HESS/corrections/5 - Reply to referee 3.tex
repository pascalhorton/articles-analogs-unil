\documentclass[]{letter}

\begin{document}

We would like to thank referee 3 for his detailed and relevant review.

Some referee comments are recalled in italics and followed by the authors’ responses. Others are addressed here, but without being recalled. The technical corrections or rephrasing are not discussed here, but will be performed.

\begin{itemize}
	
	\item Introduction: as replied to referee \#1, the introduction will be rewritten, with an improved literature review and context description. The method description will be moved to the methods section. 
	
	\item \textit{The title wrongly suggests that these conclusions are valid only in a forecasting context, while they actually have much more general implications.}
	
	We agree that the application of the MTW is not limited to forecasting. The title will be changed to the more generic term of “downscaling”. 
	
	\item The structure will be improved, with a better separation of methods, results and discussion.
	
	\item \textit{Notations: Please use conventional abbreviations for commonly used quantities: Teweles-Wobus Score $\rightarrow$ TWS or S1, continuous ranked probability (skill) score $\rightarrow$ CRP(S)S, root mean square error $\rightarrow$ RMSE}
	
	It is the journal convention (“Multi-letter variables should be avoided. Instead use single-letter variables with subscript (e.g. ERMS instead of RMSE, or ET instead of ET).”) and we were asked to change the notation of these scores this way. We agree that these are unusual notations and are very open to change back these notations if the editor say so. 
	
	\item The number of tables will be decreased, as explained to referee \#1: “We will remove some unanalyzed data, such as the spatial windows in Tables 3, 4, 7, and 8 in order to group the remaining information. Tables 10 and 11 will also be removed along with the sharpness and accuracy analysis, which is not very informative.”
	
	\item \textit{L15-17: Is it not rather because heavy precipitation events are rarer?}
	
	We cannot exclude this argument, and it might be a mix of both factors. We will add a note on that aspect.
	
	\item \textit{L24-25: I don’t understand}
	
	This reports to section 4, but might not be necessary in the abstract as it brings some confusion.
	
	\item \textit{L63: What are the parameters to calibrate? Please list them.}
	
	More details will be provided on the calibration procedure.
	
	\item \textit{L70: Please either provide a peer-reviewed reference for this decomposition or detail it here.}
	
	We will drop the analysis of the CRPS decomposition, as it brings complexity without being very informative.
	
	\item \textit{L76: Please detail the computation of the climatological distribution.}
	
	A description will be added.
	
	\item \textit{L93-94: I don’t understand.}
	
	This will be dropped as it relies on partial analysis.
	
	\item \textit{L113-114: Please justify the use of such an outdated global reanalysis (I understand this is partly for having a long time coverage). And add also the potential of using more recent and products with higher quality to the discussion.}
	
	See answer to referee \#1. These points will be addressed in the discussion. 
	
	\item \textit{L168-174: This analysis is done for different classes of precipitation values. Whether this relates to the intensity of circulation dynamics is another issue.}
	
	We will reformulate this section.
	
	\item \textit{L192-193: Figure 8 is not necessary. Please remove of put it in a supplementary material.}
	
	The figure will be removed.
	
	\item \textit{L219-220, “No relationship [: : :] criteria”: I don’t understand.}
	
	This will be removed.
	
	\item \textit{L246-250: Is it shown somewhere in the manuscript?}
	
	No it is not shown, as the figure globally is very similar to Fig. 10.
	
	\item \textit{L350-354: Given recent studies on RCM biases, I have serious doubts that RCM precipitation is reliable enough for it to be use as observed series in this context.}
	
	We agree with the referee and will change this sentence.
	
	\item Figure 11: We will drop the analysis of the CRPS decomposition, as it brings complexity without being very informative.
	
	\item Table 13: The choice for preselecting these 4 points should be somehow justified.
	You are right. These are simply the points surrounding the catchment. It will be explained.
	
	\item The other unmentioned issues will be fixed
	
	
\end{itemize}


\end{document}
