\documentclass[]{letter}

\begin{document}

The authors would like to thank referee 2 for his/her positive comments on the manuscript.

Some referee comments are recalled in italics and followed by the authors’ responses. Others are addressed here, but without being recalled. The technical corrections or rephrasing are not discussed here, but will be performed.

\begin{itemize}
	\item Sampling of extremes: The topic of extreme values within the analog method should be addressed in details, we agree on that and plan to work on it. However, it is out of the scope of this paper. The proposed MTW improvement does not change the limitation of the maximum observed values in the archive, but it is not the topic of the present paper. We will however add a note on that issue in the paper.
	
	\item Introduction and applications in hydrology: The introduction will be rewritten, as explained to referee \#1, and references will be added. Some applications to hydrology will also be cited.
	
	\item \textit{Given the focus of the paper on the sub-daily time step, I suggest to refer more on the needs in urban hydrology.}
	
	The sub-daily time step is introduced in the search of analog situations, so at the predictors level. The predictand remains a 24-h precipitation total, which limits the application to urban hydrology.
	
	\item How the sample sizes N1 and N2 are determined and optimized: Indeed, this has not been detailed here. We will provide more insight on the calibration procedure.
	
	\item The other unmentioned issues will be fixed
	
\end{itemize}


\end{document}
