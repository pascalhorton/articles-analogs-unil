%% Copernicus Publications Manuscript Preparation Template for LaTeX Submissions
%% ---------------------------------
%% This template should be used for copernicus.cls
%% The class file and some style files are bundled in the Copernicus Latex Package which can be downloaded from the different journal webpages.
%% For further assistance please contact the Copernicus Publications at: publications@copernicus.org
%% http://publications.copernicus.org


%% Please use the following documentclass and Journal Abbreviations for Discussion Papers and Final Revised Papers.


%% 2-Column Papers and Discussion Papers
\documentclass[gmd, manuscript]{copernicus}



%% Journal Abbreviations (Please use the same for Discussion Papers and Final Revised Papers)

% Geoscientific Model Development (gmd)
% Hydrology and Earth System Sciences (hess)




%% \usepackage commands included in the copernicus.cls:
%\usepackage[german, english]{babel}
%\usepackage{tabularx}
%\usepackage{cancel}
%\usepackage{multirow}
%\usepackage{supertabular}
%\usepackage{algorithmic}
%\usepackage{algorithm}
%\usepackage{float}
%\usepackage{subfig}
%\usepackage{rotating}


\begin{document}

\linenumbers

\title{AtmoSwing (v1.4): Analog Technique Model for Statistical weather forecastING}


% \Author[affil]{given_name}{surname}

\Author[1,2]{Pascal}{Horton}
\Author[1]{Michel}{Jaboyedoff}
\Author[3]{Charles}{Obled}

\affil[1]{University of Lausanne, Lausanne, Switzerland}
\affil[2]{Terranum, Lausanne, Switzerland}
\affil[3]{Universit\'{e} de Grenoble-Alpes, LTHE, Grenoble, France}

%% The [] brackets identify the author with the corresponding affiliation. 1, 2, 3, etc. should be inserted.



\runningtitle{TEXT}

\runningauthor{TEXT}

\correspondence{Pascal Horton (pascal.horton@alumnil.unil.ch)}



\received{}
\pubdiscuss{} %% only important for two-stage journals
\revised{}
\accepted{}
\published{}

%% These dates will be inserted by Copernicus Publications during the typesetting process.


\firstpage{1}

\maketitle



\begin{abstract}
% TODO: write

\end{abstract}



\introduction  %% \introduction[modified heading if necessary]
% TODO: write



\section{The analogue method}

As AtmoSwing does not rely heavily on one variant of the analogue method, but can implement different parametrization, we gathered here several references of possible implementations as well as the main evolution steps of the analogue method. This latter summarizes the improvements that were brought, as well as discontinued approaches that were found less relevant (such as the use of PCA on the predictors).

\subsection{Principles}

The analogue method is based on the principle that two similar synoptic situations may produce similar local effects \citep{Lorenz1956}. The perfect analogy does not exist, but sufficiently similar situations leading to similar effects can be identified. Thus, two states of the atmosphere that are alike are called analogues \citep{Lorenz1969}. To be relevant, this analogy must be selected by optimizing the following elements:

\begin{itemize}
	\item The analogy variable (the predictor) must contain synoptic information having indirect dependency with the local weather and the target variable (often the precipitation).
	\item The spatial window is the domain on which the analogy variable is considered. The ideal size of this area is one that maximizes useful information.
	\item The temporal window is the hour of day at which we consider the analogy variable. The analogy can be instantaneous (eg. 12~h~UTC) or averaged over a period (eg. 24~h).
	\item The analogy period is the time of year in which similar situations are being sought. Thus, we compare situations with a similar distribution of solar energy \citep{Lorenz1969}.
	\item The analogy criterion, needed to compare the variables on the chosen spatial and temporal windows, is a score used for ranking past situations according to their degree of similarity with the target situation \citep{Bontron2004}.
	\item The ideal number of analogues situations is the best compromise in order to take into account local variability and to maximize useful synoptic information  \citep{Bontron2004}.
\end{itemize}

Because of the chaotic nature of the atmosphere, two analogue situations quickly diverge over time \citep{Lorenz1969}. Thus, the method has strong limitations regarding the temporal extrapolation \citep{Bontron2004}. Numerical models being more capable of simulating the dynamic evolution of the atmosphere, the temporal extrapolation of the synoptic variables is left to them. The search for analogy aims thus at connecting the forecasted synoptic situation with a local predictand (temperature, precipitation ...), which is more difficult to simulate by numerical models.

The analogue method is being used for many years at EDF (\'{E}lectricit\'{e} de France), to build precipitation and temperature scenarios \citep{Desaint2008a}. It is also operational at the Compagnie Nationale du Rh\^{o}ne (CNR, France) and some flood forecasting services (SPC) in France and Switzerland \citep{Marty2010,GarciaHernandez2009b,Horton2012}.


\subsection{Its evolution}

\citet{Lorenz1969} introduced the concept and established the basis of the method. He considered looking for analogues on the geopotential field (at 200, 500 and 850~hPa) with a corrected Euclidean distance averaged on the 3 levels. In addition, the analogue situations were to belong to the same period of the year to be comparable in terms of distribution of solar energy.

The use of the analogue method for operational forecasting of daily precipitation originates in the work of \citet{Duband1970,Duband1974,Duband1981}. It was then designed for operational forecasting at EDF (\'{E}lectricit\'{e} de France) in order to better manage water resources and flood risks. The geopotential height at 700~hPa and 00~h was used for its higher stability than the sea level pressure (SLP) field, and its sensitivity to the atmospheric disturbances \citep{Guilbaud1997}. However, the SLP at 06~h was also used for determining the intensity of precipitation, as well as the temperature at 700~hPa as indicator of the thermal state of the air mass \citep{Duband1974}. The data were then based on 37 radiosounding over Europe and condensed by a principal component analysis. The first 25 analogues on the geopotential at 700~hPa were considered, and multiple local regressions between precipitation, SLP, and the temperature at 700~hPa were established on these analogues before being applied to the target situation \citep{Guilbaud1997}. The archives were split by season to consider comparable situations in terms of distribution of solar energy \citep{Lorenz1969}. This initial rigid division into seasons was then transformed into a moving selection of more or less two months around the target date. The forecast was then performed only on the basis of observations and was temporally extrapolated to the following two days. After a few years, this statistical temporal extrapolation was abandoned in favor of a statistical adaptation using predictors resulting from an global numerical model, allowing a forecast over the following 4~days.

Thereafter, the geopotential was considered at 1000~hPa and 700~hPa and the analogues selection was done in two stages: first according to an Euclidean distance in the space of the principal components of the geopotential field at 700~hPa, and then according to a correlation criterion (on the geopotential field at 700 and 1000~hPa, as well as on the thickness between these two levels) in order to remove days too dissimilar \citep{Guilbaud1997}. The regression approach to determining precipitation was eventually abandoned in favor of the probabilistic forecasting (synthesized by the percentiles 20, 60 and 90~\%), as it is performed today by interpolating linearly over the cumulative empirical distribution of daily precipitation measured at the analogue days \citep{Guilbaud1997,Guilbaud1998}.

\citet{Wilson1980} sought the 20 best analogues on the geopotential field at 500 and 1000~hPa using the Teweles-Wobus (S1) criterion (Eq. \ref{eq:S1}), which allows for a comparison of the gradients and thus an analogy of the atmospheric circulation. \citet{Woodcock1980} used the same criterion on the SLP fields to select 50~analogues in order to forecast maximum daily temperatures. \citet{Yacowar1975} considered the 20 best analogues on the maximum of the sum of correlations on the two fields. This work has demonstrated the best performance of the analogues method compared to regression methods, as well as the interest to search for analogues on two (500 and 1000~hPa) geopotential fields combined \citep{Guilbaud1997}.

\citet{Mandon1985} introduced a second level of analogy and assessed different variables, such as wind, humidity, air temperature, surface temperature of the Mediterranean, the product of wind and the 700~hPa moisture, as well as a second barometric criteria differentiated according to the season. \citet{Vallee1986} continued to work on the second level of analogy and considered the wind at 700~hPa \& 12~h in order to improve heavy precipitation events for a region in France susceptible to southerly flows.

Taking advantage of higher-performance IT tools \citet{Guilbaud1997} stopped using PCA in order to work directly with the raw data interpolated on grids, which was an improvement, especially when considering the S1 criterion (already used for analogues selection by\citealp{Wilson1980} and \citealp{Woodcock1980}) instead of the Euclidean distance. This score gives importance to the similarity of atmospheric circulation rather than absolute values of the pressure field. After comparing this criterion to other criteria from the literature, or combinations of these, \citet{Guilbaud1998} found that S1 is the most efficient. They also pointed out the benefit of using 2 atmospheric levels instead of one (also instead of the thickness), and to also consider 2 temporal windows \citep{Obled2002}. When using multiple predictors, they combined the criteria values calculated on each atmospheric level and each temporal window using an arithmetic mean. Finally, they also suggested using a second level of analogy on lower layers humidity and introduced the Ranked Probability Score \citep{Epstein1969} to assess the forecast quality, which was found to be more relevant that the Brier score \citep{Brier1950}.















\subsection{References implementations}
% TODO: write

\subsection{Datasets}
% TODO: write



\section{Software basics}
% TODO: write

\subsection{Structure}
% TODO: write

\subsection{Modular approach}
% TODO: write

\subsection{Technical specifications}
% TODO: write
% Language and libraries

\subsection{Performance}
% TODO: write

\subsection{Unit tests}
% TODO: write

\section{The forecaster module}
% TODO: write



\section{The viewer module}
% TODO: write




\section{The calibrator module}
% TODO: write

\subsection{Calibration framework}

The calibration of the analogue method is usually carried out in a perfect prognosis \citep{Klein1959} framework \citep{BenDaoud2010, Bontron2004}. Perfect prognosis uses observed or reanalyzed data to calibrate the relationship between predictors and predictands. Then, when used in operational forecasting, this relationship is applied to global model forecasts, that contains larger uncertainties. This framework allow us to identify relationships that are as close as possible to the natural links between predictors and predictands, by reducing uncertainties related to numerical forecasting models. However, no model is perfect, and even reanalysis data contain a bias that cannot be ignored. For this reason, the statistical relationships identified in the perfect prognosis framework should be applied to model outputs that are as similar as possible to the model used to elaborate the reanalysis. 

Another reason for working in a perfect prognosis framework is that numerical models evolve continuously, and so does the forecast they provide. Re-forecasts would allow us to work on a homogeneous dataset, as they are regularly reprocessed. However, we would need to redo the calibration procedure every time a new version is available, in order to reduce the bias \citep{Wilson2002}. Moreover, the reforecasts are not re-processed for every new version of the model, meaning we still end up with a bias between the forecast and the archive. Finally, the length of reanalyses datasets are usually much longer than reforecast datasets, which allows us to identify more robust relationships. The size of the archive is indeed an important criteria for the analogue method.

The statistical relationship is established on a calibration period that is as long as possible. For every day of this period, a search for analogues is processed, the precipitation data are associated with the corresponding dates and a forecast score is calculated. During the search for analogues situations, 120 days around the target date are excluded (thus excluding data in the same year) in order to consider only truly independent candidates days.

A validation period is always considered. It consists of an independent period that is never used as target neither candidate date. Validating the parameters of the analogue method is very important in order to avoid over-parametrization and thus to ensure that the statistical relationship is valid on another period.

The accuracy of the parameters is evaluated by means of the CRPS \citep[Continuous Ranked Probability Score,][]{Brown1974, Matheson1976, Hersbach2000}. It allows assessing the predicted cumulative distribution functions $F(y)$ compared to the observed value $y^{0}$. The better the forecast, the smaller the score. The mean CRPS of a forecast series of length $n$ can be written:

\begin{equation}
\label{eq:CRPS}
CRPS = \frac{1}{n} \sum_{i=1}^{n} \left(  \int_{-\infty}^{+\infty} \left[ F_{i}(y)-H_{i}(y-y_{i}^{0})\right]^{2} dy \right) 
\end{equation}
where $H(y-y_{i}^{0})$ is the Heaviside function that is null when $y-y_{i}^{0}<0$, and has the value 1 otherwise. The mean CRPS is averaged on the calibration, respectively the validation periods, on all days, may they be dry, slightly rainy or with heavy precipitation.

This score is now commonly used for the evaluation of continuous variables prediction systems \citep{Casati2008, Marty2010}. To compare the value of the score in regard to a reference, we often consider its skill score value, and use the climatological distribution as the reference. The CRPSS (\textit{Continuous Ranked Probability Skill Score}) is thus defined as following:

\begin{equation}
\label{eq:CRPSS}
CRPSS = \dfrac{CRPS-CRPS_{r}}{CRPS_{p}-CRPS_{r}} = 1-\dfrac{CRPS}{CRPS_{r}}
\end{equation}
where $CRPS_{r}$ is the CRPS value for the reference and $CRPS_{p}$ would be the one for a perfect forecast ($CRPS_{p} = 0$).


\subsection{Implemented forecasts scores}
% TODO: write

\subsection{The classic calibration approach}

The calibration procedure that we call ''classic'' was developed by \citet{Bontron2004} at the LTHE laboratory (INPG, Grenoble). It determines the optimal parameters for the different variables of each level of analogy. The analogy levels (eg the atmospheric circulation or humidity index) are calibrated sequentially. The procedure consists of the following steps \citep{Bontron2004}:

\begin{enumerate}
	\item Manual selection of the following parameters:
	\begin{enumerate}
		\item meteorological variable,
		\item pressure level,
		\item temporal window (hour of observation),
		\item initial analogue numbers.
	\end{enumerate}
	
	\item For every level of analogy:
	\begin{enumerate}
		\item Identification of the most skilled unitary cell (1 point for humidity variables and 4 for the geopotential height when using the S1 criteria) over a large domain. Every point (or cell) of the full domain is assessed jointly on all predictors of the level of analogy (consisting generally of the same variable, but on different pressure levels and at different hours).
		\item From this most skilled point, the spatial window is expanded by successive iterations in the direction of greater performance gain. The detailed stages are the followings: (i) The unitary spatial window is expanded in every 4 directions successively. The performance score is processed for these 4 windows. (ii) Only the direction providing the best improvement is applied to our spatial window. (iii) From this new spatial window, an increase in every 4 directions is once again assessed, and the best improvement is applied. (iv) The spatial window grows up by repeating the previous steps, until no improvement is reached.
		\item The number of analogues is optimized for the current level of analogy.
		\item A new level of analogy can be added, based on other variables (such as the humidity index) on predefined pressure levels and time frames. The number of analogues for the next level of analogy is initiated at a chosen value. Then, the procedure starts again from step (a) for the new level. The parameters calibrated on the previous analogue levels are fixed and do not change (except the number of analogues, at the final stage). 
	\end{enumerate}
	\item Finally, the numbers of analogues are re-assessed for the different levels of analogy. This is done iteratively by varying the number of analogues of each level in a systematic way.
\end{enumerate}

The calibration is thus done in successive steps. Previously calibrated parameters are generally not reassessed. The advantage of this method is that it is fast, it provides acceptable results, and it has low computing requirements. We added small improvements to this method by allowing the spatial windows to do other moves, such as: (1) increase in 2 simultaneous directions, (2) decrease in 1 or 2 simultaneous directions, (3) expansion or contraction (in every direction), (4) shift of the window (without resizing) in 8 directions (including diagonals), and finally (5) all the moves described above, but with a factor of 2, 3, or more. For example, we try to increase by 2 units in one (or more) direction. This allows to skip one size that may not be optimal.

These supplementary steps often result in spatial windows that are a bit different, but the performance gain is rather marginal. These methods are available in the open source software AtmoSwing (Analogue Technique MOdel for Statistical Weather forecastING, www.atmoswing.org).

\subsection{Monte-Carlo analysis}


\subsection{Optimization}



\section{The global optimization tools}

Genetic algorithms (GAs) come from the world of stochastic optimization, more specifically from metaheuristic approaches. These are stochastic iterative algorithms that behave like search algorithms by exploiting the characteristics of a problem and are particularly suitable for complex parameter spaces.

Genetic algorithms are part of the family of evolutionary algorithms \citep{Back1993b, Schwefel1993}, which get inspiration from some mechanisms of biological evolution, such as reproduction, genetic mutations, chromosomal crossovers, and natural selection. GAs are the most used technique from evolutionary algorithms \citep{Back1993b}, and they are constantly improving \citep{Haupt2004}. However, with time, the different methods of evolutionary algorithms tend to be similar and share many commonalities \citep{Back1996b, Haupt2004}.

The method was originally developed by \citet{Holland1992b} and popularized by \citet{Goldberg1989}. Unlike a linear or local optimization, GAs seek the global optimum on a complex surface, theoretically without restriction, but with no guarantee to reach it.


\subsection{Basic concepts of the genetic algorithms}

GAs mimic the evolution of a population of individuals in a new environment, by applying rules based on natural processes, such as DNA mutation, chromosomes crossover, natural selection, etc. It simulates the fact that in a natural environment, the most suitable individuals tend to survive longer, to reproduce more easily, and so to influence coming generations by providing genes that provide some good performance in a certain domain. Generation after generation, the DNA mixes and the strong genes cumulate in some individuals \citep{Beasley1996a}. Globally, the fitness of the population to its environment increases, while retaining enough variety to not converge too quickly to a local optimum.

Applications of GAs are very diversified as they can handle many parameters of various types \citep{Joines1996a}, even with very complex cost surfaces \citep{Haupt2004}. GAs work remarkably well with intervariable dependences \citep{Haupt2004}. The objective function to optimize (often named fitness function in this context) can be of different types (mathematical function, experimental or numerical modeling). Only the resulting value is used for optimization. Indeed, these algorithms do not require any knowledge of the problem, which can be used as black-box, but they must be adapted in order to perform optimally.

By means of the reproduction operator and the natural selection, the GAs focus on the most promising regions of the parameter space \citep{Holland1992b}. Points (parameter sets) are densified in these areas because the strong genes of the best individuals propagate from generation to generation.

Two conditions guarantee in theory the convergence to the global optimum \citep{Zitzler2004a}: (1) Parameters mutations that can allow to explore the entire parameter space, thereby ensures that any value can be achieved with a non-zero probability. (2) A rule of elitism ensuring that an optimal solution cannot be lost or damaged.

Practically, GAs allow rapidly approaching satisfactory solutions, but they do not provide the optimum solution for sure \citep{Zitzler2004a}. It is indeed mainly a matter of time. When the optimizer gets closer to the global optimum, any new improvement takes more time to appear (see figure \ref{fig:evolution}), and the final adjustment of the parameters is very time consuming \citep{Back1993a}. For problems that require a significant amount of time in order to evaluate the objective function, as in our case, we have to limit the number of generations to get reasonable processing time. Thus, different acceptable solutions can result from one or more optimizations \citep{Holland1992b}. This is a strength and a weakness of GAs: they are very good at exploring complex parameter spaces in order to identify the most promising areas, but they will not necessarily find the best solution with the optimal values of all parameters \citep{Holland1992b}.


\subsection{Structure and operators}

The GAs optimize a population of individuals (parameter sets). Each individual contains a chromosome (parameters of the analogue method in our case). We call gene every parameter that constitutes the chromosome. The parameters to optimize have long been coded in binary form and assembled as strings in the canonical GAs \citep{Goldberg1989}. Encoding and decoding steps were needed to transform the variable from its floating-point representation into its binary representation, and vice versa, which introduce quantification errors \citep{Haupt2004}. According to \citet{Holland1992b}, working with binary chromosomes was supposed to be more efficient \citep{Goldberg1990a, Back1993b}. However, more and more applications use floating-point representations, allowing to avoid the coding and decoding steps and the quantification errors \citep{Haupt2004}, and which also often resulted in a performance improvement \citep{Goldberg1990a}. It is thus now considered that for continuous variables, a floating-point representation is more suited \citep{Michalewicz1996, Herrera1998a, Haupt2004, Back1996b, Gaffney2010a}. 

There are numerous implementation variants of GAs often optimal for a given problem \citep{Hart1991a,Schraudolph1992a}. However, the structure of the method (Figure \ref{fig:structure_gas}) resulting from the work of \citet{Holland1992b} is common to most applications \citep{Back1993b}. The divergences are the operators implementation, through significantly different algorithms, which has an important effect on the results \citep{Gaffney2010a}.

All operators we used and their options, applied to real coding, are described in the following sections. Many other operators exist, but we will only present the ones we evaluated.


\subsubsection{Genesis of the population}

The first step of the optimization is to generate the initial population. A population is a set of $N$ individuals (each of which represents a point in the space of potential solutions, a parameter set of the analogue method in our application) that we are going to make evolve. A generation is the population at a given time. 

A random initialization based on a uniform distribution is the most current version. The size $N$ of the population is often a compromise between the computation time and the quality of the solution. $N$ must allow sufficient sampling of the solutions field \citep{Beasley1996a}, and should thus vary as a function of chromosome size (ie the number of parameters to be optimized). 


\subsubsection{Natural selection}

Natural selection is performed on the basis of the values of the objective function. The selection allows to only keep a certain part of the population, usually half ($N/2$), which can access the mating pool (intermediate generation with $N_{mp}$ members). If $N_{mp}$ is too high, the reproduction rate is too low, whereas if it is too small, the strong traits of individuals do not have the ability to accumulate in the same chromosome \citep{Haupt2004}. Several techniques exist, such as:

\begin{itemize}
	\item \textbf{$N_{mp}$-elitism} \citep{Michalewicz1996}: the population is sorted according to the value of the objective function and only the better half is preserved. 
	
	\item \textbf{Tournament selection} \citep{Michalewicz1996, Zitzler2004a}: two individuals are randomly selected and fight. The one with the highest score is chosen, but with a certain probability, in order to reduce the selection pressure. This procedure is repeated until the mating pool is full. Individuals can be selected several times, and thus be represented several times in the mating pool.
\end{itemize}


\subsubsection{Selection of the couples}

Individuals of the mating pool can reproduce. It begins with the selection of pairs (the parents). The techniques implemented in this work are the following:


\begin{itemize}
	\item \textbf{Rank pairing}: individuals are gathered in pairs according to their rank (classified on the performance scores). Consecutive ranks are put together (odd rows are associated with even rows). This approach is easy to achieve, but does not look like a natural process.
	
	\item \textbf{Random pairing}: two individuals are randomly selected to form a couple, according to a uniform law.
	
	\item \textbf{Roulette wheel weighting}: the roulette technique refers to gambling. But unlike casino roulette, this one is biased. Each individual is associated with a sector of the wheel with a certain opening angle, which is its probability of selection \citep{Haupt2004}. The probability assigned to the individuals is proportional to their fitness (objective function), so that the most adapted individuals have the greatest probability of reproduction. There are two techniques for weighting the individuals of the mating pool:
	
	\textit{Roulette wheel weighting on rank}: the probability of each individual depends on its rank $n$:
	\begin{equation}
	p_{n}=\dfrac{N_{mp}-n+1}{\sum^{N_{mp}}_{n=1}n}
	\label{equation_mating_rank_weighting}
	\end{equation}
	
	\textit{Roulette wheel weighting on fitness}: the selection probability is calculated based on the value of the objective function. This approach gives more weight to the best individuals when the distribution of scores is wide, while the weight is almost similar when all individuals have approximately the same score \citep{Haupt2004}. The probability $p_{n}$ of each individual is calculated by the equation \ref{equation_mating_score_weighting}:
	\begin{equation}
	p_{n}=\frac{score_{n}-score_{N_{mp}}}{\sum_{n=1}^{N_{mp}} (score_{n}-score_{N_{mp}})}
	\label{equation_mating_score_weighting}
	\end{equation}
	In our application, the last individual ($N_{mp}$) has zero probability of being selected.
	
	
	\item \textbf{Tournament selection}: This operator is similar to the one used in natural selection, but is applied here for the successive selection of each parent. To select a parent, a number of individuals (2 or 3) are randomly picked and the best is kept. This operation is performed twice, once for each partner. This approach imitates the breeding competition in nature \citep{Haupt2004}.
\end{itemize}


\subsubsection{Chromosome crossover}

Once the two parents are selected for breeding, they combine their chromosomes and produce two children, bringing the number of individuals in the population back to $N$ (the parents also return back in the total population in order to complement the next generation). The combination of chromosomes is carried out using a crossover operator, thereby generating two offspring having characteristics derived from both parents. Chromosome crossover widens the search space and favours the combination of strong genes, which can result in more suited children. It allows a mixing of genes and accumulation of positive mutations.

The evaluated crossover operators are the following:

\begin{itemize}
	\item \textbf{Single-point crossover}: a crossover point is randomly chosen for the pair. The genes (our parameters) located after that point are exchanged in between the two chromosomes.
	
	\item \textbf{Two-point crossover}: works like the single-point crossover, but there are two intersections defining the segments to exchange. This approach, which significantly extends the search space for the children, is considered more efficient than the previous \citep{Beasley1993a, Haupt2004}.
	
	\item \textbf{Multiple-point crossover} \citep{DeJong1975a}: it is a generalization of the previous, with a number of crossover points up to the number of genes.
	
	\item \textbf{Uniform crossover} \citep{Syswerda1989}: for each gene of the chromosome, it is randomly chosen to exchange or not the values between the parents.
	
	\item \textbf{Binary-like crossover} \citep{Haupt2004}: chromosome crossover on a binary coding can generate new values for variables located at intersection points, since the crossovers are applied at the bit level, thus often within a gene. This is not the case for the floating-point representation, since the crossover is performed between the genes. To reproduce the behaviour present in the original algorithms, which introduces new information, \citet{Haupt2004} propose an operator that combines standard crossover with an interpolation approach. The genes located after a crossover point are exchanged, but the gene located at the intersection is modified as follows (equation \ref{equation_mating_as_binary}):
	\begin{equation}
	\left\lbrace \begin{array}{l} 
	g_{o1,n} = g_{p1,n} - \beta (g_{p1,n} - g_{p2,n}) \\
	g_{o2,n} = g_{p2,n} + \beta (g_{p1,n} - g_{p2,n}) \\
	\end{array} \right.
	\label{equation_mating_as_binary}
	\end{equation}
	where $g_{o1,n}$ and $g_{o2,n}$ are the $n$-th gene of the two new offspring, and $g_{p1,n}$ and $g_{p2,n}$ are those of the two parents. $\beta$ is a random value between 0 and 1.
	
	\item \textbf{Blending method} \citep{Radcliffe1991a}: in this approach, instead of exchanging the genes in between the chromosomes after one or multiple crossover points, these are combined by linear combination (equation \ref{equation_mating_blending_method}). The genes of the parents are blended together using a random value ($\beta$) that can be unique for the whole chromosome, or that can change for every gene. The genes of the offspring are bounded by the genes of the parents, no value can be out of their range.
	\begin{equation}
	\left\lbrace \begin{array}{l} 
	g_{o1,n} = \beta g_{p1,n} + (1-\beta)g_{p2,n} \\ 
	g_{o2,n} = (1-\beta) g_{p1,n} + \beta g_{p2,n} \\
	\end{array} \right.
	\label{equation_mating_blending_method}
	\end{equation}
	
	\item \textbf{Linear crossover} \citep{Wright1991a}: in order to allow the genes to take values outside the interval defined by the parents, a method of extrapolation is necessary. Linear crossover introduces such an approach, and produces three children from two parents, following equation \ref{equation_mating_linear_crossover}. Less couples are required in order to fill up the generation.
	\begin{equation}
	\left\lbrace \begin{array}{l} 
	g_{o1,n} = 0.5 g_{p1,n} + 0.5 g_{p2,n} \\ 
	g_{o2,n} = 1.5 g_{p1,n} - 0.5 g_{p2,n} \\ 
	g_{e3,n} = - 0.5 g_{p1,n} + 1.5 g_{p2,n} \\ 
	\end{array} \right.
	\label{equation_mating_linear_crossover}
	\end{equation}
	
	\item \textbf{Heuristic crossover} \citep{Michalewicz1996}: it is a variation of the latter methods that relies on the following equation:
	\begin{equation}
	\left\lbrace \begin{array}{l} 
	g_{o1,n} = \beta (g_{p1,n} - g_{p2,n}) + g_{p1,n} \\
	g_{o2,n} = \beta (g_{p2,n} - g_{p1,n}) + g_{p2,n} \\
	\end{array} \right.
	\label{equation_mating_heuristic_crossover}
	\end{equation}
	
	\item \textbf{Linear interpolation}: unlike previous techniques, this technique does not rely on crossover points, but on a linear interpolation on every gene of the couple (equation \ref{equation_mating_linear_interpolation}).
	\begin{equation}
	\left\lbrace \begin{array}{l} 
	c_{o1} = c_{p1} - \beta (c_{p1} - c_{p2}) \\
	c_{o2} = c_{p2} + \beta (c_{p1} - c_{p2}) \\
	\end{array} \right.
	\label{equation_mating_linear_interpolation}
	\end{equation}
	where $c_{o1}$ and $c_{o2}$ are the full chromosomes of the offspring, and $c_{p1}$ an $c_{p2}$ are the ones of the parents. As before, $\beta$ is a random value between 0 and 1, and is here the same for every gene.
	
	\item \textbf{Free interpolation}: this technique performs interpolation on each gene, like the previous one; but in this case, the weighting factor changes for each gene:
	\begin{equation}
	\left\lbrace \begin{array}{l} 
	c_{o1} = c_{p1} - [\beta_{1} (g_{p1,1} - g_{p2,1}), \beta_{2} (g_{p1,2}\\
	~~~~~~~~~~~~ - g_{p2,2}), ..., \beta_{Ng} (g_{p1,N_{g}} - g_{p2,N_{g}})] \\
	c_{o2} = c_{p2} + [\beta_{1} (g_{p1,1} - g_{p2,1}), \beta_{2} (g_{p1,2}\\
	~~~~~~~~~~~~ - g_{p2,2}), ..., \beta_{Ng} (g_{p1,N_{g}} - g_{p2,N_{g}})] \\
	\end{array} \right.
	\label{equation_mating_free_interpolation}
	\end{equation}
	where $N_{g}$ is the number of genes, and $\beta$ is here independent between the genes.
	
\end{itemize}

Many other methods or variations exist, combining the advantages of different approaches. The performance of the variants being related to the problem to be addressed, we can not identify a priori the best technique for our application.


\subsubsection{Mutation}

The combination of strong genes by the operator of chromosomes crossover is theoretically the most important operating mechanism in the conventional GAs \citep{Holland1992b,Back1993b}. However, many studies identify the mutation process as main operator, and crossovers as secondary \citep[see][]{Back1992a,Back1996a,Back1996b,Smith1997a,Deb1999,Haupt2004,Costa2005a,Costa2007a}.

The mutation operator is a direct modification of genes. In a binary coding, it is implemented as an inversion of some bits in a chromosome, while in real coding, it is done by changing the gene values. Mutations add diversity to the population and prevent a freeze of the evolution, or a genetic drift to a local optimum. Thus, it makes the convergence to the global optimum theoretically possible \citep{Beasley1993a}, as they allow exploring beyond the current region of the parameter space. They therefore help preventing the algorithm to converge too quickly to a local optimum and bring new characteristics that were not present in the original population \citep{Haupt2004}. 

The evaluated and developed mutation operators are the following:

\begin{itemize}
	\item \textbf{Uniform mutation}: The mutation rate is constant and equal for every gene of each individual; they all have the same probability to mutate. When a gene is selected for mutation, a new random value is assigned, according to a uniform law.
	
	\item \textbf{Variable uniform mutation} \citep{Fogarty1989}: a variable mutation rate over the generations was first suggested by \citet{Holland1992b} and evaluated by \citet{Fogarty1989}. It improved significantly the performance of GAs. In most applications, the mutation rate decreases with the generations, in a deterministic and global (for all individuals) manner \citep{Back1992b}. Its optimum configuration depends on the size of the chromosomes, of the properties of the objective function, and of the population size \citep{Back1992b}. We implemented this operator according to equation \ref{equation_mutation_uniformvariable}.
	\begin{equation}
	p_{n,G} = p_{G_{0}}+\left( \dfrac{p_{G_{0}}-p_{G_{m,p}}}{G_{m,p}} \right) min\left\lbrace G,G_{m,p}\right\rbrace 
	\label{equation_mutation_uniformvariable}
	\end{equation}
	where $p_{n,G}$ is the mutation rate (probability) of the gene $n$ for generation number $G$, $G_{m,p}$ is the maximum number of generations during which the mutation rate varies. $p_{G_{0}}$ is the initial mutation probability, and $p_{G_{m,p}}$ is the final one. $p_{G_{0}}$, $p_{G_{m,p}}$ and $G_{m,p}$ are the three controlling parameters of the operator. The evolution of the mutation rate is linear.
	
	\item \textbf{Constant normal mutation}: many use normal distributions to generate new values. The gene $g$ that mutate becomes:
	\begin{equation}
	g' = N(g,\sigma^{2})
	\label{equation_mutating_normal_distribution}
	\end{equation}
	where $\sigma$ is the standard deviation of the distribution. The disadvantage of this technique is that an accurate value of $\sigma$ must be chosen \citep{Haupt2004}, which is impossible to know beforehand.
	
	\item \textbf{Variable normal mutation}: with the same logic that the variable uniform mutation, we tested a mutation operator using a normal distribution with a variable mutation rate and standard deviation. The mutation rate is calculated with equation \ref{equation_mutation_uniformvariable}. On the same principle, we decrease linearly the standard deviation over the generations:
	\begin{equation}
	\sigma_{n,G} = \sigma_{G_{0}}+\left( \dfrac{\sigma_{G_{0}}-\sigma_{G_{m,\sigma}}}{G_{m,\sigma}} \right) min\left\lbrace G,G_{m,\sigma}\right\rbrace 
	\label{equation_mutation_normalvariable}
	\end{equation}
	where $\sigma_{n,G}$ is the standard deviation of gene $n$ et generation number $G$, $\sigma_{G_{0}}$ is the initial standard deviation, $\sigma_{G_{m,\sigma}}$ is the final standard deviation, $G_{m,\sigma}$ is the maximum number of generations during which the standard deviation varies. $p_{G_{0}}$, $p_{G_{m,p}}$, $G_{m,p}$, $\sigma_{G_{0}}$, $\sigma_{G_{m,\sigma}}$ and $G_{m,\sigma}$ are the six parameters of the method.
	
	\item \textbf{Non-uniform mutation} \citep{Michalewicz1996}: two random numbers are picked based on a uniform law: $r_{1}$, which determines the direction of the change, and $r_{2}$, which determines its magnitude. The new value of the gene is given by the following equation:
	\begin{equation}
	g_{n}^{'} = 
	\left\lbrace \begin{array}{l l} 
	g_{n} + \left(b_{n}-g_{n}\right) r_{2} \left(1 - \dfrac{G}{G_{m}} \right)^{2} & if \; r_{1} < 0.5 \\
	g_{n} - \left(g_{n}-a_{n}\right) r_{2} \left(1 - \dfrac{G}{G_{m}} \right)^{2} & if \; r_{1} \geq 0.5 \\
	\end{array} \right.
	\label{equation_mutation_nonuniform_original}
	\end{equation}
	where $a_{n}$ is the is the lower bound of the $n$-th gene, $b_{n}$ its upper bound, $G$ the present generation, and $G_{m}$ the maximum number of generations.
	
	We adapted this operator for our application, which is not based on a predefined number of generations:
	\begin{equation}
	g_{n}^{'} = 
	\left\lbrace \begin{array}{l l} 
	g_{n} + \left(b_{n}-g_{n}\right) r_{2} \left(1 - \min \left\lbrace \dfrac{G}{G_{m,r}}, 1 \right\rbrace \left(1-\omega\right) \right)^{2} & if \; r_{1} < 0.5 \\
	g_{n} - \left(g_{n}-a_{n}\right) r_{2} \left(1 - \min \left\lbrace \dfrac{G}{G_{m,r}}, 1 \right\rbrace \left(1-\omega\right) \right)^{2} & if \; r_{1} \geq 0.5 \\
	\end{array} \right.
	\label{equation_mutation_nonuniform}
	\end{equation}
	where $G_{m,r}$ is the maximum number of generations during which the magnitude of the research varies, and $\omega$ is a threshold chosen by the user to maintain a minimum search radius when $G>G_{m,r}$. During the first generations, the exploration extent covers the entire parameter space. However, this area is reduced over generations, allowing exploitation of local solutions.
	
	\item \textbf{Individual adaptive mutation rate} \citep{Back1992a}: based on the ideas of Evolution Strategies \citep[see][]{Rechenberg1973, Schwefel1981}, \citet{Back1992a} introduced a concept of self-adaptive genetic algorithms. The idea is to distribute control parameters within individuals themselves, which partially decentralize control of the evolution. It allows reducing the parametrization of GAs and introducing a notion of self-management. The first approach is the introduction of a mutation rate per individual, that mutates itself under its own probability \citep{Back1992a}. Then, the eventual new rate is used to mutate the genes of the individual. Thus, as this rate decreases, it will have less probability of being itself mutated. This approach is close to the natural adaptation phenomena. A population less suited to its environment is changing faster than better adapted populations. Mutations are performed according to a constant uniform distribution. The initial mutation rates are randomly chosen \citep{Back1992a} and the method has no parameter. Other approaches exist to introduce a self-adaptation \citep[see][]{Smith1997a,Deb1999,Deb2001a}.
	
	\item \textbf{Individual adaptive search radius}: based on the ideas of the non-uniform mutation, we introduce a search radius in the approach of individual adaptive mutation rates. This search radius $r_{a}$, bounded between 0 and 1, is also adaptive and behaves similarly to the adaptive mutation rates. In order to separate its evolution from the one of the mutation rate, its own value is considered initially as a self-mutation rate to eventually mutate before being used as a normalized search radius. The value of a mutated gene is given by the following equation, which is a simplification of the non-uniform mutation:
	\begin{equation}
	g_{n}^{'} = 
	\left\lbrace \begin{array}{l l} 
	g_{n} + \left(b_{n}-g_{n}\right) r_{2} r_{a} & if \; r_{1} < 0.5 \\
	g_{n} - \left(g_{n}-a_{n}\right) r_{2} r_{a} & if \; r_{1} \geq 0.5 \\
	\end{array} \right.
	\label{equation_mutation_rayon_adaptatif}
	\end{equation}
	where $r_{1}$ and $r_{2}$ are randomly selected, in the same way as for the non-uniform mutation. No external parameter is therefore necessary.
	
	\item \textbf{Chromosome of adaptive mutation rate} \citep[\textit{n adaptative mutation rate},][]{Back1992a}: analogously to the individual adaptive mutation rate, this approach leaves the control of the evolution rate to the individuals themselves. The difference here is that each gene has a specific mutation rate. The main advantage is that the tuning of the mutation can be much more precise \citep{Smith1997a}. We therefore consider a second chromosome containing the mutation rate for each gene of the first chromosome. The operations of mutation and self-mutation are similar to the case of the individual adaptive mutation rate, but in a distributed way, within the chromosome. Another difference is that we apply the same crossover operations as those applied to the first chromosome, and this for the same crossing points. Thus, during an exchange of genes, children also inherit the mutation rates specific for each of these genes.
	
	\item \textbf{Chromosome of adaptive search radius}: we introduced this operator that combines the operations of the chromosome of adaptive mutation rate to our adaptive search radius approach. Similarly, an individual has 3 chromosomes: the first containing the values to be optimized, the second containing the distributed mutation rate, and the last one, the distributed search radius. Again, no external parameters are required.
	
	\item \textbf{Multi-scale mutation}: finally, we developed another approach, that is also based on the search radius concept. However, the latter is not decreasing with time. Methods based on a reduction of the mutation rate or radius simulate a transition from the exploration phase to the exploitation one. The idea is consistent as long as we are confident that the algorithm will converge towards the global optimum. Indeed, once the algorithm is in the exploitation mode, it is very unlikely to go out of the region it converges to. We wanted to test an approach that combines both exploration and exploitation during the whole optimization. Thus, we considered the search radius $r_{a}$ of equation \ref{equation_mutation_rayon_adaptatif} as a random value for each individual, but restricted to 4 equiprobable values: 1, 0.5, 0.1, 0.02. The only external parameter is the mutation rate which is fixed.
	
	
\end{itemize}

When the gene to mutate is represented by a list of distinct values (eg meteorological variable or analogy criterion), the random choice of a new value is always based on a uniform distribution, without notion of search radius. There is indeed no meaning to use operators based on principles of proximity when the latter does not exist.


\subsubsection{Elitism}

We used a process of elitism on the natural selection as well as on mutations. This ensures the survival of the best individual so that we do not lose a better solution. This approach is very common in the field of GAs \citep{Haupt2004}. After the natural selection operator, if the best individual has not been selected, it is copied to the mating pool instead of an individual randomly picked. After mutation, if the best individual has mutated and if its new version has a lower score than the original, the latter is reinserted instead of an individual randomly chosen.


\subsubsection{Ending the optimization}

The convergence check determines whether the solution is acceptable and if the algorithm may stop. The stopping criteria are not often well documented in GAs case studies. We chose to stop the optimization if the best individual does not change for $x$ generations. This value should not be too low to allow the algorithm to escape from a local optima. In addition, the rate of improvement decreases with the progression of the optimization. It is thus common that the best individual does not evolve over several generations when we get closer to the global solution. We chose a value of $x=20$ generations.


\subsection{Implementation and constraints}

Some constraints need to be taken into account. For example, when a crossover or a mutation operation results in a parameter value standing out of the authorized bounds, it has to be brought back within the limits. Moreover, the parameters are of different nature: some are continuous, such as the weight, some are discretized, such as the analogues number, or the spatial windows, and finally, some are independent elements in an array, such as the selection of the meteorological variable. New values resulting from the optimizer need to respect the type of data it represents.

Other constraints exist in between the parameters, such as the temporal window of the humidity index that has to be consistent in between the relative humidity and the precipitable water. Another example is the weighting of the different pressure levels which has to be normalized.

GAs are very computationally intensive because they require many evaluations of the objective function. These assessments are very long in our application, as they require calculating and assessing a forecast for every day of the calibration period. In order to reduce the computation time, we avoid recalculating the score of an individual who has previously been evaluated and that has not changed. We keep the score of each individual living in the selection until it mutates.

The assessment (calculation of the objective function) of each member of the population of a generation is completely independent and can be performed in parallel on different processors of a computer \citep{Alliot2005}. We implemented this technique and the resulting time savings was very important. In order to perform optimizations for multiple time series, the use of a cluster is a necessity, which our code allows.


\subsection{Recommendations of parametrization}

The GAs parametrization, such as the mutation rate, population size, natural selection options, and so on, is difficult given the high number of existing variants, each developed for a specific problem \citep{Haupt2004, Costa2007a}. This parametrization depends on the objective function, implementation variants, the range of the parameters to be optimized, and performance indicators. Thus, different studies suggest very different parametrization.

A key element of the parametrization of GAs is finding the right balance between exploration and exploitation \citep{Back1992a, Smith1997a}. Exploration is characterized by a relatively high probability to assess the regions of the parameter space that have not yet been visited. This probability must be sufficiently large at the beginning of the optimization, so that the algorithm is capable of identifying the region where the global optimum is located. Exploitation is characterized by a local search in an area of interest, and generally makes small movements. The latter is interesting to refine the results at the end of the optimization.

\citet{DeJong1975a} and \citet{Grefenstette1986} compared different implementations and parametrizations of GAs on functions of varying complexity. They observed that a small population size improves the initial performance, while a large population improves long-term performance. They also observed that the ratio of the population to keep for the mating pool is around 50\% (45\% to 60\%).

Values of the mutation rate varies broadly between the studies: from 0.001 \citep{DeJong1975a} to 0.2 \citep{Haupt2004}. \citet{Back1996b} showed that mutation rates higher than the usual ranges are more optimal at the beginning of the optimization, allowing further exploration. The combination of a small population and a high mutation rate works best for the first generations \citep{DeJong1975a, Back1996b, Haupt2004}, but as we could observe, it does not guarantee the quality of the final result. Incremental approaches with varying mutation rates are certainly more optimal but more complex to implement \citep{Back1996a, Back1996b}.


\subsubsection{Comparison process and results}

One of our goals being to make recommendations of parametrization in order to optimize the analogue method, we proceeded systematically. The results are summarized hereafter \citep[see][for the details]{Horton2012a}. We used concepts from the factorial design approach \citep[see eg.][]{Costa2005a,Costa2007a,Mariano2010a}, which is sometimes used for comparative analysis of different parametrizations of GAs. We processed by stages, analysing in details and in a systematic way every variants of the implemented operators, in combination with multiple other options and parameters in order to take into account eventual co-dependencies. 

In order to evaluate a combination of operators/options, we processed 10 optimizations for one parametrization of GAs. The performances were characterized by four indicators:

\begin{itemize}
	\item mean score: average of the final scores of the 10 optimizations,
	\item convergence: the number of optimizations that converged to a supposed global optimum,
	\item number of generations: characterization of the convergence speed,
	\item number of evaluations of the objective function: characterization of the required calculation time (more realistic than the number of generations).
\end{itemize}

This comparison required tens of thousands of optimizations that were performed on a cluster of the University of Lausanne. The results, detailed in \citet{Horton2012a} are synthesized hereafter:

\begin{itemize}
	\item \textbf{Population size}: we found the following ranges to be accurate in average:
	
	$50<N<100$ for a very simple implementation of the analogue method (1 level of analogy with 2 pressure levels),
	
	$N\approx200$ for a bit more complex method (1 level of analogy with 4 pressure levels, or 2 level of analogy with less pressure levels),
	
	$N\approx500$ for significantly more complex methods (2-3 levels of analogy with 4 pressure levels for the atmospheric circulation, and 2 to 4 levels for the humidity analogy),
	
	We didn't find any improvement with $N>500$, the results were even surprisingly of a lower quality. However, this cannot be generalized and depends on the analogue method to optimize, and supposedly on the characteristics of the processes generating the precipitations in a given region. 
	
	\item \textbf{Natural selection}: this operator has no significant influence, and both tested implementations work fine, with a slightly better performance for the ratio-elitism.
	
	\item \textbf{Selection of couples}: 6 variants of the couples selection were assessed. The performance of these variants are relatively close, both in terms of score, convergence, and number of evaluations. The random pairing performed the most poorly, when the tournament selection with 3 candidates was slightly superior. The roulette wheel weighting is not far behind, but it is less effective in terms of convergence and number of evaluations. This operator has not a significant role in our application. 
	
	\item \textbf{Chromosomes crossover}: we compared 21 different options of the crossover operators. This analysis revealed some slightly better options, some bad ones, and many averages. Among the bad operators, we find first the heuristic crossover, which is also more demanding in number of evaluations, as well as the linear crossover. Binary-like crossovers (especially with 2 points of intersection, whether $\beta$ is shared or not) are significantly better than the others, especially in terms of convergence. The two points crossover is relatively close. Other operators can be considered usable, yet may not be optimal. Once again, this operator is not the key of the GAs parametrization.
	
	\item \textbf{Mutation}: we compared the 10 mutation operators with different options, bringing the number of variations of this operator up to 110. We immediately observed that the mutation operator has a very important role on the performance of the optimizations of the analogue method, and that the other reproduction operators seem of secondary importance. This observation is in line with the work of \citet{Back1996a}, who argues for the importance of mutation over reproduction. He even suggests, in opposition with the theory of \citet{Holland1992a}, that chromosomes crossovers have mostly a corrective role of mutation operations. Various studies have also identified the importance of the mutation operator relatively to reproduction \citep[see eg.][]{Back1992a, Back1996b, Smith1997a, Deb1999, Haupt2004, Costa2005a, Costa2007a}.
	
	The mutation operators based on a variable normal or uniform law work very poorly and are difficult to configure. We then observe many operators more or less with the same scores and requiring a variable amount of assessments. The convergence analysis allows us to highlight three best operators:
	
	\textit{Non-uniform mutation} \citep{Michalewicz1996}: this operator is good in terms of convergence, mainly when the number of parameters to optimize is rather low. The number of required evaluations, however, can be quite substantial. The main disadvantage of the non-uniform mutation is the complexity of its parametrization, which is difficult to estimate a priori. These parameters must be carefully chosen to be in line with the evolution rate of the population, and are therefore dependent on the problem being treated. We could observe that the $\omega$ coefficient does not influence performance. The role of $G_{m}$ is rather difficult to judge, but does not seem essential. The mutation rate was found to be important. The difficulty is that the optimal value seems to be very case-related. Indeed, by even changing the precipitation time series (ie optimizing for another subregion), but not the complexity of the analogue method, the optimal mutation rate changes, making it impossible to estimate in advance.
	
	\textit{Chromosome of adaptive search radius}: unlike the previous one, our new operator is very robust, as it requires no option and is auto-adapting. It may be sometimes a little bit slower for simple problems, but does not require parametrization, which is an important advantage. It is interesting to notice that our insertion of an extra chromosome representing the search radius gives better performance than other self-adaptive operators (such as, for example, the chromosome of adaptive mutation rate).
	
	\textit{Multi-scale mutation}: finally, our multi-scale mutation, which also performs pretty well, can as well be seen as fairly robust, since it requires only one parameter, the mutation rate. However, it can also be difficult to estimate a correct value a priori.
	
	It may be wise to perform multiple optimizations and to consider these three operators in parallel in order to obtain results from algorithms that are either sometimes more efficient or more robust. It is interesting to note that the three best techniques incorporate a notion of search distance. It is likely that this notion is the key to these algorithms, for our application, and allows them to initially explore the parameter domain, and then to converge. The search radius in fact directly represents the notion of transition between exploration and exploitation, in our opinion more than a possible evolution of mutation rates.
	
	\item \textbf{Other options}: A ratio of 50\% for the mating pool seems to be a good choice.
	
\end{itemize}


\subsubsection{Recommended parametrization of GAs}

We evaluated the optimization by genetic algorithms on methods of varying complexity, with a large number of combinations of operators to be able to make recommendations for optimizing the analogue method. Our conclusions are:

\begin{itemize}
	\item The optimization does not systematically converge to the global optimum  (but still often nearby), which is why we recommend doing several optimizations in parallel in order to compare the results, analyse the convergence, and keep the best.
	
	\item The population size should be in accordance with the complexity of the method to optimize: from 50 for the simple ones, up to 500 for the most complex methods.
	
	\item The value of the ratio for the intermediate population is not so important, and value of 50\% seems quite appropriate.
	
	\item Ratio-elitism is slightly better than tournaments for the natural selection operator, but it is not decisive.
	
	\item The performance of the operators for the couples selection perform relatively similarly. The roulette wheel weighting and the tournament selection are more efficient in terms of convergence and required number of evaluations.
	
	\item Most crossover operators have relatively similar performance. Binary-like crossover with two points of intersection are better than others, especially for convergence.
	
	\item Mutation has a clearly dominant influence. Three mutation operators stand out, two of which we have developed: the non-uniform mutation, the multi-scale mutation, and the chromosome of adaptive search radius. The latter is the most robust as it has no controlling parameter.
	
\end{itemize}

In order to make an optimization of the analogue method with genetic algorithms, it may be wise to consider these three mutation operators in parallel. We would then combine algorithms that are sometimes faster to other that are more robust. In order to be confident in the optimized methods, we propose using a set of the following mutation operators:

\begin{itemize}
	\setlength\itemsep{-4px}
	\item 1x non-uniform, $p_{mut}=0.1$, $G_{m}=50$, $\omega=0.1$
	\item 1x non-uniform, $p_{mut}=0.1$, $G_{m}=100$, $\omega=0.1$
	\item 1x non-uniform, $p_{mut}=0.2$, $G_{m}=100$, $\omega=0.1$
	\item 1x multi-scale,  $p_{mut}=0.1$
	\item 2x chromosome of adaptive search radius
\end{itemize}



\conclusions  %% \conclusions[modified heading if necessary]
% TODO: write




\appendix
\section{}    %% Appendix A

\subsection{}                               %% Appendix A1, A2, etc.




\begin{acknowledgements}
Thanks to Hamid Hussain-Khan of the University of Lausanne for his help and availability, and for the intensive use of the cluster he is in charge of. Thanks to Renaud Marty for his fruitful collaboration over the years.

Thanks to the Swiss Federal Office for Environment (FOEV), the Roads and Water courses Service, Energy and Water Power Service of the Wallis Canton and the Water, Land and Sanitation Service of the Vaud Canton who financed the MINERVE project which started this research. NCEP reanalysis data provided by the NOAA/OAR/ESRL PSD, Boulder, Colorado, USA, from their Web site at http://www.esrl.noaa.gov/psd/. Precipitation time series provided by MeteoSwiss. 
\end{acknowledgements}


%% REFERENCES

%% The reference list is compiled as follows:

%\begin{thebibliography}{}
%
%\bibitem[AUTHOR(YEAR)]{LABEL}
%REFERENCE 1
%
%\bibitem[AUTHOR(YEAR)]{LABEL}
%REFERENCE 2
%
%\end{thebibliography}

\bibliographystyle{copernicus}
\bibliography{../_refs/_articles-atmoswing}

%% Since the Copernicus LaTeX package includes the BibTeX style file copernicus.bst,
%% authors experienced with BibTeX only have to include the following two lines:
%%
%% \bibliographystyle{copernicus}
%% \bibliography{example.bib}
%%
%% URLs and DOIs can be entered in your BibTeX file as:
%%
%% URL = {http://www.xyz.org/~jones/idx_g.htm}
%% DOI = {10.5194/xyz}


%% LITERATURE CITATIONS
%%
%% command                        & example result
%% \citet{jones90}|               & Jones et al. (1990)
%% \citep{jones90}|               & (Jones et al., 1990)
%% \citep{jones90,jones93}|       & (Jones et al., 1990, 1993)
%% \citep[p.~32]{jones90}|        & (Jones et al., 1990, p.~32)
%% \citep[e.g.,][]{jones90}|      & (e.g., Jones et al., 1990)
%% \citep[e.g.,][p.~32]{jones90}| & (e.g., Jones et al., 1990, p.~32)
%% \citeauthor{jones90}|          & Jones et al.
%% \citeyear{jones90}|            & 1990



%% FIGURES

%% ONE-COLUMN FIGURES

%%f
%\begin{figure}[t]
%\includegraphics[width=8.3cm]{FILE NAME}
%\caption{TEXT}
%\end{figure}
%
%%% TWO-COLUMN FIGURES
%
%%f
%\begin{figure*}[t]
%\includegraphics[width=12cm]{FILE NAME}
%\caption{TEXT}
%\end{figure*}
%
%
%%% TABLES
%%%
%%% The different columns must be seperated with a & command and should
%%% end with \\ to identify the column brake.
%
%%% ONE-COLUMN TABLE
%
%%t
%\begin{table}[t]
%\caption{TEXT}
%\begin{tabular}{column = lcr}
%\tophline
%
%\middlehline
%
%\bottomhline
%\end{tabular}
%\belowtable{} % Table Footnotes
%\end{table}
%
%%% TWO-COLUMN TABLE
%
%%t
%\begin{table*}[t]
%\caption{TEXT}
%\begin{tabular}{column = lcr}
%\tophline
%
%\middlehline
%
%\bottomhline
%\end{tabular}
%\belowtable{} % Table Footnotes
%\end{table*}
%
%
%%% NUMBERING OF FIGURES AND TABLES
%%%
%%% If figures and tables must be numbered 1a, 1b, etc. the following command
%%% should be inserted before the begin{} command.
%
%\addtocounter{figure}{-1}\renewcommand{\thefigure}{\arabic{figure}a}
%
%
%%% MATHEMATICAL EXPRESSIONS
%
%%% All papers typeset by Copernicus Publications follow the math typesetting regulations
%%% given by the IUPAC Green Book (IUPAC: Quantities, Units and Symbols in Physical Chemistry,
%%% 2nd Edn., Blackwell Science, available at: http://old.iupac.org/publications/books/gbook/green_book_2ed.pdf, 1993).
%%%
%%% Physical quantities/variables are typeset in italic font (t for time, T for Temperature)
%%% Indices which are not defined are typeset in italic font (x, y, z, a, b, c)
%%% Items/objects which are defined are typeset in roman font (Car A, Car B)
%%% Descriptions/specifications which are defined by itself are typeset in roman font (abs, rel, ref, tot, net, ice)
%%% Abbreviations from 2 letters are typeset in roman font (RH, LAI)
%%% Vectors are identified in bold italic font using \vec{x}
%%% Matrices are identified in bold roman font
%%% Multiplication signs are typeset using the LaTeX commands \times (for vector products, grids, and exponential notations) or \cdot
%%% The character * should not be applied as mutliplication sign
%
%
%%% EQUATIONS
%
%%% Single-row equation
%
%\begin{equation}
%
%\end{equation}
%
%%% Multiline equation
%
%\begin{align}
%& 3 + 5 = 8\\
%& 3 + 5 = 8\\
%& 3 + 5 = 8
%\end{align}
%
%
%%% MATRICES
%
%\begin{matrix}
%x & y & z\\
%x & y & z\\
%x & y & z\\
%\end{matrix}
%
%
%%% ALGORITHM
%
%\begin{algorithm}
%\caption{�}
%\label{a1}
%\begin{algorithmic}
%�
%\end{algorithmic}
%\end{algorithm}
%
%
%%% CHEMICAL FORMULAS AND REACTIONS
%
%%% For formulas embedded in the text, please use \chem{}
%
%%% The reaction environment creates labels including the letter R, i.e. (R1), (R2), etc.
%
%\begin{reaction}
%%% \rightarrow should be used for normal (one-way) chemical reactions
%%% \rightleftharpoons should be used for equilibria
%%% \leftrightarrow should be used for resonance structures
%\end{reaction}
%
%
%%% PHYSICAL UNITS
%%%
%%% Please use \unit{} and apply the exponential notation


\end{document}
