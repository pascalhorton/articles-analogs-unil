%% Version 4.3.2, 25 August 2014
%
%%%%%%%%%%%%%%%%%%%%%%%%%%%%%%%%%%%%%%%%%%%%%%%%%%%%%%%%%%%%%%%%%%%%%%
% Template.tex --  LaTeX-based template for submissions to the 
% American Meteorological Society
%
% Template developed by Amy Hendrickson, 2013, TeXnology Inc., 
% amyh@texnology.com, http://www.texnology.com
% following earlier work by Brian Papa, American Meteorological Society
%
% Email questions to latex@ametsoc.org.
%
%%%%%%%%%%%%%%%%%%%%%%%%%%%%%%%%%%%%%%%%%%%%%%%%%%%%%%%%%%%%%%%%%%%%%
% PREAMBLE
%%%%%%%%%%%%%%%%%%%%%%%%%%%%%%%%%%%%%%%%%%%%%%%%%%%%%%%%%%%%%%%%%%%%%

%% Start with one of the following:
% DOUBLE-SPACED VERSION FOR SUBMISSION TO THE AMS
\documentclass{ametsoc}

% TWO-COLUMN JOURNAL PAGE LAYOUT---FOR AUTHOR USE ONLY
% \documentclass[twocol]{ametsoc}

%%%%%%%%%%%%%%%%%%%%%%%%%%%%%%%%
%%% To be entered only if twocol option is used

\journal{jcli}

%  Please choose a journal abbreviation to use above from the following list:
% 
%   jamc     (Journal of Applied Meteorology and Climatology)
%   jtech     (Journal of Atmospheric and Oceanic Technology)
%   jhm      (Journal of Hydrometeorology)
%   jpo     (Journal of Physical Oceanography)
%   jas      (Journal of Atmospheric Sciences)	
%   jcli      (Journal of Climate)
%   mwr      (Monthly Weather Review)
%   wcas      (Weather, Climate, and Society)
%   waf       (Weather and Forecasting)
%   bams (Bulletin of the American Meteorological Society)
%   ei    (Earth Interactions)

%%%%%%%%%%%%%%%%%%%%%%%%%%%%%%%%
%Citations should be of the form ``author year''  not ``author, year''
\bibpunct{(}{)}{;}{a}{}{,}

%%%%%%%%%%%%%%%%%%%%%%%%%%%%%%%%

%%% To be entered by author:

%% May use \\ to break lines in title:

\title{Global Optimization of the Analogue Method by Means of Genetic Algorithms}

%%% Enter authors' names, as you see in this example:
%%% Use \correspondingauthor{} and \thanks{Current Affiliation:...}
%%% immediately following the appropriate author.
%%%
%%% Note that the \correspondingauthor{} command is NECESSARY.
%%% The \thanks{} commands are OPTIONAL.

    %\authors{Author One\correspondingauthor{Author One, 
    % American Meteorological Society, 
    % 45 Beacon St., Boston, MA 02108.}
% and Author Two\thanks{Current affiliation: American Meteorological Society, 
    % 45 Beacon St., Boston, MA 02108.}}

\authors{Pascal Horton\correspondingauthor{Terranum, Rue de l'industrie 35 bis, 1030 Bussigny-pr\`{e}s-Lausanne, Switzerland.}}

%% Follow this form:
    % \affiliation{American Meteorological Society, 
    % Boston, Massachusetts.}

\affiliation{University of Lausanne, Lausanne, and Terranum,  Bussigny-pr\`{e}s-Lausanne, Switzerland}

%% Follow this form:
    %\email{latex@ametsoc.org}

\email{pascal.horton@terranum.ch}

%% If appropriate, add additional authors, different affiliations:
    %\extraauthor{Extra Author}
    %\extraaffil{Affiliation, City, State/Province, Country}

\extraauthor{Michel Jaboyedoff}
\extraaffil{University of Lausanne, Lausanne, Switzerland}

%% May repeat for a additional authors/affiliations:

%\extraauthor{}
%\extraaffil{}

\extraauthor{Charles Obled}
\extraaffil{Universit\'{e} de Grenoble-Alpes, LTHE, Grenoble, France}

%%%%%%%%%%%%%%%%%%%%%%%%%%%%%%%%%%%%%%%%%%%%%%%%%%%%%%%%%%%%%%%%%%%%%
% ABSTRACT
%
% Enter your abstract here
% Abstracts should not exceed 250 words in length!
%
% For BAMS authors only: If your article requires a Capsule Summary, please place the capsule text at the end of your abstract
% and identify it as the capsule. Example: This is the end of the abstract. (Capsule Summary) This is the capsule summary. 

\abstract{Enter the text of your abstract here.}

\begin{document}

%% Necessary!
\maketitle


%%%%%%%%%%%%%%%%%%%%%%%%%%%%%%%%%%%%%%%%%%%%%%%%%%%%%%%%%%%%%%%%%%%%%
% MAIN BODY OF PAPER
%%%%%%%%%%%%%%%%%%%%%%%%%%%%%%%%%%%%%%%%%%%%%%%%%%%%%%%%%%%%%%%%%%%%%
%

%% In all cases, if there is only one entry of this type within
%% the higher level heading, use the star form: 
%%
% \section{Section title}
% \subsection*{subsection}
% text...
% \section{Section title}

%vs

% \section{Section title}
% \subsection{subsection one}
% text...
% \subsection{subsection two}
% \section{Section title}

%%%
% \section{First primary heading}

% \subsection{First secondary heading}

% \subsubsection{First tertiary heading}

% \paragraph{First quaternary heading}


\section{Introduction}
...
\cite{Horton2012b}


\section{Calibration of the analogue method}
...

\subsection{Data}
...

\subsection{The analogue method}
...

\cite{Horton2012}

\subsection{Calibration framework}

The calibration of the analogue method is usually done in a perfect prognosis \citep{Klein1959a} framework \citep{BenDaoud2010a, Bontron2004a}. Perfect prognosis uses observed or reanalyzed data to calibrate the relationship between predictors and predictands. Then, when used in operational forecasting, this relationship is applied to global model forecasts, that contains larger uncertainties. This framework allow us to identify relationships that are as close as possible to the natural links between predictors and predictands, by reducing uncertainties related to numerical forecasting models. However, no model is perfect, and even reanalysis data contains a bias that cannot be ignored. For this reason, the statistical relationships identified in the perfect prognosis framework should be applied to model outputs that are as similar as possible to the model used to elaborate the reanalysis. 

Another reason for working in a perfect prognosis framework is that numerical models evolve continuously, and so does the forecast they provide. Re-forecasts allows us to work on a homogeneous dataset, as they are regularly reprocessed. However, it would involve that we need to redo the calibration procedure every time a new version is available, in order to reduce the bias \citep{Wilson2002a}. Moreover, the reforecasts are not re-processed for every new version of the model, meaning we still end up with a bias between the forecast and the archive. Finally, the length of reanalyses datasets are usually much longer than reforecast datasets, which allows us to identify more robust relationships. The size of the archive is indeed an important criteria for the analogue method.

The statistical relationship is established on a calibration period that is as long as possible. For every day of this period, a search for analogues is processed, the precipitation data are associated with the corresponding dates and a forecast score is calculated. During the search for analogues situations, 120 days around the target date are excluded (thus excluding data in the same year) in order to consider only truly independent candidates days.

A validation period is always considered. It consists of an independent period that is never used as target neither candidate date. Validating the parameters of the analogue method is very important in order to avoid over-parametrization and thus to ensure that the statistical relationship is valid on another period.

The accuracy of the parameters is evaluated by means of the CRPS \citep[Continuous Ranked Probability Score,][]{Brown1974, Matheson1976, Hersbach2000}. Let the precipitation variable be denoted $x$ with $x^{0}$ the observed value, and $F(x)$ the predicted cumulative distribution functions (cdf). The mean CRPS of a forecast series of length $n$ can be written:

\begin{equation}
\label{eq:CRPS}
CRPS = \frac{1}{n} \sum_{i=1}^{n} \left(  \int_{-\infty}^{+\infty} \left[ F_{i}(x)-H_{i}(x-x_{i}^{0})\right]^{2} dx \right) 
\end{equation}
where $H(x-x_{i}^{0})$ is the Heaviside function that is null when $x-x_{i}^{0}<0$, and has the value 1 otherwise.

The mean CRPS is processed on the calibration, respectively the validation periods. It means that we average the scores on all days, may they be dry, slightly rainy or with heavy precipitation. There is actually a weighting by the number: climatologically, non-rainy days are often more frequent (of course, depending on the location) and it is essential to forecast them well. The days with heavy precipitation are rare, but can instead provide individual scores ($CRPS_{j}$) very penalizing, which implies that the optimization will also try to forecast them accurately. However, these aspects are not explicitly controlled.


\subsection{The classic calibration approach}

The calibration procedure that we call ''classic'' was developed by \citet{Bontron2004a} at the LTHE laborytory (INPG, Grenoble). It determines the optimal settings for the different variables of each level of analogy. The analogy levels (eg the atmospheric circulation or humidity variables) are calibrated sequentially. The procedure consists of the following steps \citet{Bontron2004a}:

\begin{enumerate}
	\item Manual choice of the following parameters:
	\begin{itemize}
		\item meteorological variable,
		\item atmospheric level,
		\item time frame (hour of observation),
		\item initial analogue numbers.
	\end{itemize}
	
	\item For every level of analogy:
	\begin{enumerate}
		\item Identification of the most skilled unitary cell (1 point for humidity variables and 4 for the geopotential fields when using the S1 criteria) over a large domain. Every point (or cell) of the full domain is assessed jointly on every predictor of the level of analogy (consisting generally of the same variable, but on different atmospheric levels and at different hours).
		\item From this most skilled point, the spatial window is expanded by successive iterations in the direction of greater performance gain. The detailed stages are the followings:
		\begin{enumerate}
			\item The unitary spatial window is expanded in every 4 directions successively. The performance score is processed for these 4 windows.
			\item Only the direction providing the best improvement is applied to our spatial window.
			\item From this new spatial window, an increase in every 4 directions is once again assessed, and the best improvement is applied.
			\item The spatial window grows up by repeating the previous steps, until no improvement is reached.
		\end{enumerate}
		\item The number of analogues is optimized for the current level of analogy.
		\item A new level of analogy can be added, based on other variables on predefined atmospheric levels and time frames. The analogue number for the next level of analogy is initiated at a chosen value. Then, the procedure starts again from step (a). The parameters calibrated on the previous analogue levels are fixed and do not change (except the number of analogues, at the final stage). 
	\end{enumerate}
	\item Finally, the numbers of analogues are re-assessed for the different analogue levels. This is done iteratively by varying the number of analogues of each level in a systematic way.
\end{enumerate}

Calibration is done in stages, to determine the optimal parameters systematically. The steps are distinct and previously optimized parameters are generally not reassessed. \citet{Bontron2004a} notes however that ''this type of algorithm, which is changing the parameters of a model in a unique path, can lead to the best solution, provided that there is no local optima''. The advantage of this method is that it is fast and has low computing requirements. We added small improvements to this method by allowing the spatial windows to do other moves, such as:

\begin{itemize}
	\item increase in 2 simultaneous directions,
	\item decrease in 1 or 2 simultaneous directions,
	\item expansion or contraction (in every direction),
	\item shift of the window (without resizing) in 8 directions (including diagonals),
	\item and finally all the moves described above, but with a factor of 2, 3, or more. For example, we try to increase by 2 units in one (or more) direction. This allows to skip one size that may not be optimal.
\end{itemize}

These supplementary steps often result in spatial windows that are a bit different, but the performance gain is rather marginal.











\subsection{Limits of the classic calibration}
...


\section{Characterization of the parameters space}
...

\subsection{Monte-Carlo analysis}
...

\subsection{Parameters that should be optimized}
...

\subsection{Possible new degrees of freedom}
...


\section{Adapting the genetic algorithms}
...

\subsection{Basic concepts of the genetic algorithms}
...

\subsection{Structure and operators}
...

\subsection{Implementation and contraints}
...

\subsection{Recommendations of parameterization}
...


\section{Optimization of the analogy of atmospheric circulation}
...

\subsection{Which parameters are optimized ?}
...

\subsection{Results of the optimization of atmospheric circulation}
...

\subsection{Discussion}
...


\section{Optimization of the analogy with humidity variables}
...

\subsection{Results of the optimization on both analogy levels}
...

\subsection{Successive optimization vs global optimization}
...

\subsection{Discussion}
...


\section{Conclusions and perspectives}
...



%%%%%%%%%%%%%%%%%%%%%%%%%%%%%%%%%%%%%%%%%%%%%%%%%%%%%%%%%%%%%%%%%%%%%
% ACKNOWLEDGMENTS
%%%%%%%%%%%%%%%%%%%%%%%%%%%%%%%%%%%%%%%%%%%%%%%%%%%%%%%%%%%%%%%%%%%%%
%
\acknowledgments
Start acknowledgments here.

%%%%%%%%%%%%%%%%%%%%%%%%%%%%%%%%%%%%%%%%%%%%%%%%%%%%%%%%%%%%%%%%%%%%%
% APPENDIXES
%%%%%%%%%%%%%%%%%%%%%%%%%%%%%%%%%%%%%%%%%%%%%%%%%%%%%%%%%%%%%%%%%%%%%
%
% Use \appendix if there is only one appendix.
%\appendix

% Use \appendix[A], \appendix}[B], if you have multiple appendixes.
%\appendix[A]

%% Appendix title is necessary! For appendix title:
%\appendixtitle{}

%%% Appendix section numbering (note, skip \section and begin with \subsection)
% \subsection{First primary heading}

% \subsubsection{First secondary heading}

% \paragraph{First tertiary heading}

%% Important!
%\appendcaption{<appendix letter and number>}{<caption>} 
%must be used for figures and tables in appendixes, e.g.,
%
%\begin{figure}
%\noindent\includegraphics[width=19pc,angle=0]{figure01.pdf}\\
%\appendcaption{A1}{Caption here.}
%\end{figure}
%
% All appendix figures/tables should be placed in order AFTER the main figures/tables, i.e., tables, appendix tables, figures, appendix figures.
%
%%%%%%%%%%%%%%%%%%%%%%%%%%%%%%%%%%%%%%%%%%%%%%%%%%%%%%%%%%%%%%%%%%%%%
% REFERENCES
%%%%%%%%%%%%%%%%%%%%%%%%%%%%%%%%%%%%%%%%%%%%%%%%%%%%%%%%%%%%%%%%%%%%%
% Make your BibTeX bibliography by using these commands:
% \bibliographystyle{ametsoc2014}
% \bibliography{references}

\bibliographystyle{ametsoc2014}
% \bibliography{references}
\bibliography{../_refs/_articles-optimization}

%%%%%%%%%%%%%%%%%%%%%%%%%%%%%%%%%%%%%%%%%%%%%%%%%%%%%%%%%%%%%%%%%%%%%
% TABLES
%%%%%%%%%%%%%%%%%%%%%%%%%%%%%%%%%%%%%%%%%%%%%%%%%%%%%%%%%%%%%%%%%%%%%
%% Enter tables at the end of the document, before figures.
%%
%
%\begin{table}[t]
%\caption{This is a sample table caption and table layout.  Enter as many tables as
%  necessary at the end of your manuscript. Table from Lorenz (1963).}\label{t1}
%\begin{center}
%\begin{tabular}{ccccrrcrc}
%\hline\hline
%$N$ & $X$ & $Y$ & $Z$\\
%\hline
% 0000 & 0000 & 0010 & 0000 \\
% 0005 & 0004 & 0012 & 0000 \\
% 0010 & 0009 & 0020 & 0000 \\
% 0015 & 0016 & 0036 & 0002 \\
% 0020 & 0030 & 0066 & 0007 \\
% 0025 & 0054 & 0115 & 0024 \\
%\hline
%\end{tabular}
%\end{center}
%\end{table}

%%%%%%%%%%%%%%%%%%%%%%%%%%%%%%%%%%%%%%%%%%%%%%%%%%%%%%%%%%%%%%%%%%%%%
% FIGURES
%%%%%%%%%%%%%%%%%%%%%%%%%%%%%%%%%%%%%%%%%%%%%%%%%%%%%%%%%%%%%%%%%%%%%
%% Enter figures at the end of the document, after tables.
%%
%
%\begin{figure}[t]
%  \noindent\includegraphics[width=19pc,angle=0]{figure01.pdf}\\
%  \caption{Enter the caption for your figure here.  Repeat as
%  necessary for each of your figures. Figure from \protect\cite{Knutti2008}.}\label{f1}
%\end{figure}

\end{document}